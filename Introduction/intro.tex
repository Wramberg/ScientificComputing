\pagenumbering{arabic}
\setcounter{page}{1}

\chapter{Theoretical aspects} \label{ch:intro}
In this chapter, theoretical aspects of the translational loop on \figref{fig:assignmentloop} are presented and discussed.

\begin{figure}[h!]
    \centering
    \begin{pspicture}(9,10)(0,5.75)
    %\psgrid
        \psset{arrowscale=1.5,fillstyle=solid,fillcolor=lightgray} %linecolor=gray

        %wires
        \psline{->}(0,9)(1,9)   \rput(0,9.25){x}
        \psline{->}(2,9)(3,9)   \rput(2.5,9.25){u}
        \psline{->}(5,9)(6,9)   \rput(5.5,9.25){v}
        \psline{->}(7,9)(9,9)   \rput(9,9.25){y}
        \psline[arrowscale=1.25]{*-}(8,9)(8,7)
        \psline{->}(8,7)(7,7)
        \psline{->}(6.5,6)(6.5,6.5)    \rput(6.5,5.75){z}
        \psline{->}(6,7)(5,7)   \rput(5.5,7.25){j}
        \psline(3,7)(1.5,7)
        \psline{->}(1.5,7)(1.5,8.5)     \rput(1.25,7.75){s}

        %lpf
        \psset{linecolor=gray}
        \psframe(3,9.5)(5,8.5)  \rput(4,9){LPF}

        %bpf
        \psframe(3,7.5)(5,6.5)  \rput(4,7){BPF}

        %mixers
        \pscircle(1.5,9){0.5}   \rput(1.5,9){\gray \LARGE$\times$}
        \pscircle(6.5,9){0.5}   \rput(6.5,9){\gray \large X} % OSCILLATOR
        \pscircle(6.5,7){0.5}   \rput(6.5,7){\gray \LARGE$\times$}
    \end{pspicture}

    \caption{Translational loop}
    \label{fig:assignmentloop}
\end{figure}

\section{Mathematical expressions}
In this section, mathematical expressions are derived for the different signals in the loop. If the numerically controlled oscillator (NCO) is considered first, the output $y(t_n)$ can be expressed as
\begin{flalign}
& & y(t_n) &= 1 \cdot \cos\big(2\pi f_0 t_n + \phi'_y(t_n)\big) & & \label{eq:yfunction}
\end{flalign}
where $f_0$ denotes the free running frequency, $t_n$ is discrete time-steps given as $\frac{n}{f_s}$ for $n=1,2,...,N-1$ and $\phi'_y(t_n)$ is given as
\begin{flalign}
& & \phi'_y(t_n) &= 2\pi k_{co} \cdot \frac{1}{f_s} \sum\limits_{\ell=0}^{n}v(t_{\ell}) & & \label{eq:phiy}
\end{flalign}
where $k_{co}$ is the NCO constant, $v(t_{\ell})$ is the NCO input and $f_s$ denotes the sampling frequency. The point of the translational loop is to control the phase of the output $\phi'_y(t_n)$, or rather, have it follow the phase of the input. To do this, the frequency difference between the in- and output must first be accounted for. This is done by translating the free-running frequency $f_0$ to the input frequency $f_{\text{if}}$. As shown on the figure, this is done by mixing the oscillator output with an external signal. This signal is given as
\begin{flalign}
& & z(t_n) &= 2\cos\big(2\pi(f_0-f_{\text{if}})t_n\big) & &
\end{flalign}
where, as seen, the frequency is equal to the NCO frequency with the wanted output frequency subtracted. Note that this means the external oscillator signal is low-side injected (in terms of frequency translation). This means that image (or mirror) frequency will be located at $f_0-2\cdot f_{\text{if}}$ which can interfere with the loop. Since an ideal case is considered however, the image frequency will not be taken into consideration. The output signal of the mixing $j(t_n)$ can be expressed as
\begin{flalign}
& & j(t_n) &= z(t_n) \cdot y(t_n) = 2\cos\big(2\pi(f_0-f_{\text{if}})t_n\big)      \cdot   \cos\big(2\pi f_0 t_n + \phi'_y(t_n)\big) & & \\
& & &= \cos\big(2\pi(f_0-f_{\text{if}}+f_0)t_n + \phi'_y(t_n)\big)   +  \cos\big(2\pi(f_0-f_{\text{if}}-f_0)t_n - \phi'_y(t_n)\big)  & & \\
& & &= \cos\big(2\pi(2f_0-f_{\text{if}})t_n + \phi'_y(t_n)\big)   +  \cos\big(2\pi(-f_{\text{if}})t_n - \phi'_y(t_n)\big)  & &
\end{flalign}
where the mixing product located at $f_{\text{if}}$ is the one of interest. For this reason the bandpass (BP) filter should be located around $f_{\text{if}}$ yielding the $s(t_n)$ signal which can be rewritten as
\begin{flalign}
& & s(t_n) &= \cos\big(2\pi(-f_{\text{if}})t_n - \phi'_y(t_n)\big)   & &  \\
& &        &= \cos\big(2\pi f_{\text{if}} \:t_n + \phi'_y(t_n)\big)  & & \label{eq:sfunc}
\end{flalign}
assuming an ideal BP filter which completly removes the other mixing product. There are (at least) two important aspects of the $s(t_n)$ signal that should be noted. First, the frequency of the signal is equal to that of the input signal which is important when the phases are to be compared. Second, the $s(t_n)$ signal has the same phase as the NCO output which is essential since it simplifies the comparison. To compare the NCO output phase with the phase of the input signal, the input signal is mixed with the $s(t_n)$ signal. The input signal $x(t_n)$ is given as
\begin{flalign}
& & x(t_n) &= 1 \cdot \sin\big(2\pi f_{\text{if}} \: t_n + \phi'_x(t_n)\big) & &
\end{flalign}
which means the mixing output can be expressed as
\begin{flalign}
& & u(t_n) &= x(t_n) \cdot s(t_n) & & \\
& & u(t_n) &= \sin\big(2\pi f_{\text{if}} \: t_n + \phi'_x(t_n)\big)    \cdot     \cos\big(2\pi f_{\text{if}} \:t_n + \phi'_y(t_n)\big)     & & \\
& & u(t_n) &= \frac{1}{2}\bigg( \sin\big(2\pi 2f_{\text{if}} \: t_n + \phi'_x(t_n) + \phi'_y(t_n) \big)    +    \sin\big(\phi'_x(t_n) - \phi'_y(t_n)\big)\bigg)     & & \label{eq:uneq}
\end{flalign}
which results in the lower mixing product only depending on the phase difference. To isolate this, the $u(t_n)$ signal is lowpass filtered which means $v(t_n)$ can be expressed as 
\begin{flalign}
& & v(t_n) &= \frac{1}{2}\sin\big(\phi'_x(t_n) - \phi'_y(t_n)\big)     & &
\end{flalign}
assuming perfect lowpass filtering. Note that the sine operation (not cosine) is used in this expression to ensure that the output is equal to zero when there is no phase difference. This causes the system to be non-linear, however, due to the sine function.

\section{Initial state considerations}
As stated in the assignment, there is no way to know the initial state for the system. Due to feedback and filtering in the loop, however, initial values for $n=-1$ must be assumed for some the signals.
\newpage
Assuming that 
\begin{flalign}
& & \phi'_x(t_{-1}) &= \phi'_y(t_{-1})\big) = 0 & &
\end{flalign}
$u(t_{-1})$ can be found using \eqref{eq:uneq} to
\begin{flalign}
& & u(t_{-1}) &= \frac{1}{2}\bigg( \sin\big(2\pi 2f_{\text{if}} \: t_{-1} + \phi'_x(t_{-1}) + \phi'_y(t_{-1}) \big)    +    \sin\big(\phi'_x(t_{-1}) - \phi'_y(t_{-1})\big)\bigg)     & & \\
& &           &= \frac{1}{2}\bigg( \sin\big(2\pi 2f_{\text{if}} \: t_{-1} \big)\bigg)     & & 
\end{flalign}
Now assuming that $t_{-1} = 0 $, this yields
\begin{flalign}
& & u(t_{-1}) &= 0 = v(t_{-1}) & &
\end{flalign}
Which corresponds with \eqref{eq:phiy} where $v(t_{-1})$ must equal zero for the initial phase $\phi'_y(t_{-1})$ to be zero. This in turn means that $s(t_{-1})$, following \eqref{eq:sfunc}, must be given as
\begin{flalign}
& & s(t_{-1}) &= \cos\big(2\pi f_{\text{if}} \:t_{-1} + \phi'_y(t_{-1})\big) \\
& &           &= \cos\big(2\pi f_{\text{if}} \:0)\big) \\
& &           &= 1 & &
\end{flalign}
Which means the initial state for the loop can be given as
\begin{flalign}
& & s(t_{-1}) &= 1 \\
& & u(t_{-1}) &= 0 \\
& & v(t_{-1}) &= 0 & &
\end{flalign}
These values could of course be chosen differently, as long as they respect the underlying equations.

\section{Phase transfer function}
For small phase differences between $x(t_n)$ and $s(t_{n-1})$ the following holds
\begin{flalign}
%& & \phi'_x(t_n) - \phi'_y&(t_{n-1}) < 0.1  & & \\
%& & &\Downarrow & & \notag \\
& & \frac{1}{2}\sin\big(\phi'_x(t_n) - \phi'_y(t_{n-1})\big)  &= \frac{1}{2}(\phi'_x(t_n) - \phi'_y(t_{n-1}))   & &
\end{flalign}

Which can be utilized to express the output phase $\phi'_y(t_n)$ as
\begin{flalign}
& & \phi'_y(t_n) &= 2\pi k_{co} \cdot \frac{1}{f_s} \sum\limits_{\ell=0}^{n}v(t_{\ell}) = 2\pi k_{co} \cdot \frac{1}{f_s} \left( \sum\limits_{\ell=0}^{n-1}v(t_{\ell}) + v(t_n) \right) & & \\
%& &              &= 2\pi k_{co} \cdot \frac{1}{f_s} \left( \sum\limits_{\ell=0}^{n-1}v(t_{\ell}) + v(t_n) \right) & & \\
& &              &= 2\pi k_{co} \cdot \frac{1}{f_s} \left( \sum\limits_{\ell=0}^{n-1}v(t_{\ell}) + \phi'_x(t_n) - \phi'_y(t_{n-1}) \right) & & \label{eq:phaseaprox}
\end{flalign}
\newpage
Where the summation can be expressed as
\begin{flalign}
& & \phi'_y(t_{n-1}) &= 2\pi k_{co} \cdot \frac{1}{f_s} \sum\limits_{\ell=0}^{n-1}v(t_{\ell}) & & \\
& &\Downarrow & & &  \notag \\ 
& & \sum\limits_{\ell=0}^{n-1}v(t_{\ell}) &= \frac{f_s}{2\pi k_{co}} \phi'_y(t_{n-1})   & & \label{eq:summationexp}
\end{flalign}
Inserting \eqref{eq:summationexp} into \eqref{eq:phaseaprox} yields that
\begin{flalign}
& & \phi'_y(t_n) &= 2\pi k_{co} \cdot \frac{1}{f_s} \left( \sum\limits_{\ell=0}^{n-1}v(t_{\ell}) + \phi'_x(t_n) - \phi'_y(t_{n-1}) \right) & & \\
& &              &= 2\pi k_{co} \cdot \frac{1}{f_s} \left( \frac{f_s}{2\pi k_{co}} \phi'_y(t_{n-1}) + \phi'_x(t_n) - \phi'_y(t_{n-1}) \right) & & \\
& &              &= \phi'_y(t_{n-1}) +  2\pi k_{co} \cdot \frac{1}{f_s} \phi'_x(t_n) -  2\pi k_{co} \cdot \frac{1}{f_s} \phi'_y(t_{n-1}) & &
\end{flalign}
Where isolating the phase terms on each side of the equation yields
\begin{flalign}
& & 2\pi k_{co} \cdot \frac{1}{f_s} \phi'_x(t_n) &= \phi'_y(t_n) - \phi'_y(t_{n-1}) + 2\pi k_{co} \cdot \frac{1}{f_s} \phi'_y(t_{n-1})& & \\
& & &\Downarrow \mathcal Z(\cdot) & & \notag \\
& & 2\pi k_{co} \cdot \frac{1}{f_s} \phi'_x(z) &= \phi'_y(z) - \phi'_y(z)z^{-1} + 2\pi k_{co} \cdot \frac{1}{f_s} \phi'_y(z) z^{-1}& & \\
& & 2\pi k_{co} \cdot \frac{1}{f_s} \phi'_x(z) &= \phi'_y(z) \left(1 - z^{-1} + 2\pi k_{co} \cdot \frac{1}{f_s} z^{-1} \right)& &
\end{flalign}
From which the phase transfer function can be expressed as
\begin{flalign}
& & \phi'(z) = \frac{\phi'_y(z)}{\phi'_x(z)} & = \frac{2\pi k_{co} \cdot \frac{1}{f_s}}{ 1 - z^{-1} + 2\pi k_{co} \cdot \frac{1}{f_s} z^{-1}} & & \\
& &                               & = \frac{2\pi k_{co} \cdot \frac{1}{f_s}}{ 1 - (1 - 2\pi k_{co} \cdot \frac{1}{f_s}) z^{-1}} & &
\end{flalign}
%Setting $k_{co}=500\cdot10^3$ and $fs=17.33 \cdot 10^6$ yields
%\begin{flalign}
%& & \phi'(z) = \frac{\phi'_y(z)}{\phi'_x(z)} &= \frac{2\pi 500\cdot10^3 \cdot \frac{1}{17.33 \cdot 10^6}}{ 1 - (1 - 2\pi 500\cdot10^3 \cdot \frac{1}{17.33 \cdot 10^6}) z^{-1}} \\
%& &                               &= \frac{0.1812}{1-0.8188z^{-1}} & &
%\end{flalign}


\chapter{Simulation platform} \label{ch:intro}
In this chapter, a simulation platform is developed to perform tests on the translational loop. This platform is developed in MATLAB, just kidding, in Python. To do this, the numpy (np), scipy.signal (signal) and matplitlib.pyplot (plt) Python modules are used. It should be noted that the \texttt{\_\_future\_\_} statement for division is also used to map the division operator \texttt / to the \texttt{\_\_truediv\_\_()} function to ensure float division.
\section{Input}
To import the input data, the function presented on \cref{code:input} was developed.

\lstset{language=python,caption=Function for importing input data,label=code:input}
\begin{lstlisting}
def getinput():
    global N, fif, fs

    # Load first N samples of I and Q from text files
    I = np.genfromtxt("gmsk_I.dat")[:N]
    Q = np.genfromtxt("gmsk_Q.dat")[:N]

    # Get oversampling ratio and symbol rate, calculate sample frequency
    os    = np.genfromtxt("gmsk_os.dat")
    fsymb = np.genfromtxt("gmsk_fsymb.dat")
    fs    = fsymb * os

    # Generate time axis and input signal x
    tn = np.arange(len(I)) / fs
    x = I * np.cos(2 * np.pi * fif * tn) - Q * np.sin(2 * np.pi * fif * tn)

    # Calculate phase of x
    phix = np.unwrap(np.arctan2(Q, I))

    return tn, x, phix
\end{lstlisting}
As seen, the function uses the globally defined N as the number of samples to import. Furthermore, it uses a globally defined \texttt{fif} and \texttt{fs} to generate the input signal \texttt{x}. The function returns the time vector \texttt{tn} aswell as the phase \texttt{phix} and the signal itself \texttt{x}. After importing the input, the platform simulates the translational loop sample by sample, which is described in the next section.


\section{Translational loop}
To simulate the translation loop, the function presented on \cref{code:transloop} was developed.\\

\lstset{language=python,caption=Function for simulating tranlational loop,label=code:transloop}
\begin{lstlisting}
def loop(x, tn):
    global fs, f0, fif, kco, N

    # Values of s[-1], u[-1] and v[-1]
    sm1 = 1
    um1 = 0
    vm1 = 0

    # Get coefficients of 1st order Butterworth lowpass filter
    b, a = signal.butter(1, 200e3 / (fs / 2), 'low')

    # Initialize arrays for signals
    u    = np.empty(N)
    s    = np.empty(N)
    v    = np.empty(N)
    phiy = np.empty(N)

    # Calculate signals for n = 0
    u[0]    = x[0] * sm1
    v[0]    = u[0] * b[0] + um1 * b[1] - vm1 * a[1]
    phiy[0] = 0
    s[0]    = 0

    # Main loop
    for n in range(1, N):
        u[n]    = x[n] * s[n - 1]
        v[n]    = u[n] * b[0] + u[n - 1] * b[1] - v[n - 1] * a[1]
        phiy[n] = 2 * np.pi * kco * 1 / fs * sum(v[:n])
        s[n]    = np.cos(2 * np.pi * fif * tn[n] + phiy[n])

    return u, v, phiy, s
\end{lstlisting}

As seen, the function takes the input signal and the time-vector as input. Subsequently, it simulates the translational loop sample-by-sample and returns relevant signals. The output signal y is not returned but can be computed from the phase signal \texttt{phiy} as shown in the next section. To compare the phase signals a function was developed for plotting them which is presented on \cref{code:phaseplot}.

\lstset{language=python,caption=Function for plotting phase signals,label=code:phaseplot}
\begin{lstlisting}
def plotphases(phix, phiy, phiycomp=None):
    plt.figure(figsize=(11,2.75), dpi=200)   
    plt.plot(phix, label='Phase of input signal')
    plt.plot(phiy, label='Phase of output signal')
    if phiycomp is not None:
        plt.plot(phiycomp, label='Compensated phase of output signal')
    plt.grid()
    plt.xlabel('Samples')
    plt.legend(loc=3, prop={'size': 12}, labelspacing=0.25)
    plt.subplots_adjust(left=0.1, right=0.9, bottom=0.175)
    if phiycomp is not None:
        plt.savefig('phasesignalscomp.eps') 
    else:
        plt.savefig('phasesignals.eps') 
\end{lstlisting}


\section{Output}
To compute the output signal from the phase signal, the function presented on \cref{code:output} can be used.

\lstset{language=python,caption=Function for generation output signal,label=code:output}
\begin{lstlisting}
def gety(phiy, tn):
    global f0
    y = np.cos(2 * np.pi * f0 * tn + phiy)
    return y
\end{lstlisting}


\section{Phase error}
To compute the phase error between the in- and output phases, the function in \cref{code:phaseerr} was developed. This function returns the error signal as well as the root mean square (RMS) error.

\lstset{language=python,caption=Function for computing phase errors,label=code:phaseerr}
\begin{lstlisting}
def getphaseerror(phix, phiy):
    try:
        diff = phix[:phiy.size] - phiy
    except:
        raise Exception('phiy.size must be < than phix.size')
    err = np.sqrt(np.mean(diff[250:] ** 2))
    return diff, err
\end{lstlisting}

To plot the phase error, the function in \cref{code:phaseerrplot} was developed.
\lstset{language=python,caption=Function for plotting phase errors,label=code:phaseerrplot}
\begin{lstlisting}
def plotphaseerr(x):
    plt.figure(figsize=(11,2.75), dpi=200)  
    plt.plot(x)
    plt.grid()
    plt.xlabel('Samples')
    plt.subplots_adjust(left=0.1, right=0.9, bottom=0.175)
    plt.savefig('phaseerr.eps')  
\end{lstlisting}


\section{Phase compensation}
To perform phase compensation on the output signal, the function in \cref{code:phasecomp} was developed. This function minimizes the RMS error (between the phase signals) to find the time-offset and the average error to find the phase offset.\\

\lstset{language=python,caption=Function for performing phase compensation,label=code:phasecomp}
\begin{lstlisting}
def phasecomp(phix, phiy):
    # Find phase offset
    diff = phix[250:] - phiy[250:]
    poff = np.mean(diff)

    # Find time offset
    maxoffset = 50
    rmserr = np.empty(maxoffset)
    for i in np.arange(0, maxoffset):
        dummy, rmserr[i] = getphaseerror(phix[200:], phiy[200+maxoffset-i:]+poff)

    toff = maxoffset - np.argmin(rmserr)
    if toff < 0:
        raise Exception('Got negative time offset, time delay should be positive')

    return phiy[toff:] + poff
\end{lstlisting}

To find the time-offset a cross-correlation approach was initially considered. For time-delay estimation the cross correlation approach shows poor performance in high-snr applications however \cite{193195}. For this reason, the RMS approach was taken instead which is very similar to the average square difference estimator discussed in \cite{193195}. The only difference is the squareroot which has no effect on the resulting time-delay.
%Note that \texttt{signal.fftconvolve()} is used instead of \texttt{np.correlate()}. This was chosen because convolution in frequency domain is considerably faster. Testing with 30.000 samples yields a \SI{33.6}{\milli\second} difference as shown on \cref{ode:fftconvcorr}.

%\lstset{language=python,caption=Time tests for fft convolve vs. standard correlation,label=code:fftconvcorr}
%\begin{lstlisting}
%In [1]: run sim.py
%In [2]: tn , x , phix = getinput()   
%In [3]: u , v , phiy , s = loop(x, tn)
%In [4]: diff = phix[250:] - phiy[250:]
%In [5]: poff = np.mean(diff)
%In [6]: %timeit signal.fftconvolve(phix, phiy[-500:500:-1]+poff, mode='valid')
%100 loops, best of 3: 11.6 ms per loop
%In [7]: %timeit np.correlate(phix, phiy[500:-500]+poff)
%10 loops, best of 3: 45.2 ms per loop
%\end{lstlisting}


\section{Averaged periodogram}
To compute the averaged periodogram of a signal, the function in \cref{code:periodo} was developed.

\lstset{language=python,caption=Function for computing averaged periodogram,label=code:periodo}
\begin{lstlisting}
def getperiodogram(x):
    global fs
    freqs, psd = signal.welch(x, fs=fs, window='boxcar', nperseg=300, noverlap=None, detrend='constant', return_onesided=True)
    return freqs, psd
\end{lstlisting}

As seen, the \texttt{signal.welch()} function is used to compute the PSD. Natively, this function computes an averaged PSD by slicing overlapping segments of the input signal, windowing, detrending, and finally performing FFT, on each. In this simulation platform, however, the overlapping and windowing has been omitted and the detrending only removes DC-offset. This means the function simply computes an averaged periodogram using 300 samples non-overlapping segments. To present the PSD, the function returns a frequency axis and the computed PSD values along this axis. This is used by the function in \cref{code:periodoplot} for plotting periodograms.

\lstset{language=python,caption=Function for plotting periodogram of in- and output signals,label=code:periodoplot}
\begin{lstlisting}
def plotpsd(x, y):
    freqsx, psdx = getperiodogram(x)
    freqsy, psdy = getperiodogram(y)
    plt.figure(figsize=(11,2.75), dpi=200)  
    plt.plot(freqsx/1e3, psdx, label='PSD of input signal')
    plt.plot(freqsy/1e3, psdy, label='PSD of output signal')
    plt.grid()
    plt.xlim(0, 3.5e3)
    plt.xlabel('Frequency [kHz]')
    plt.legend(loc=1, prop={'size': 12}, labelspacing=0.25)
    plt.subplots_adjust(left=0.1, right=0.9, bottom=0.175)
    plt.savefig('psds.eps') 
\end{lstlisting}

\chapter{Simulation results} \label{ch:intro}
In this chapter, results are presented from the simulation platform developed in the previous chapter.
 
\section{Power spectrum density of in- and output (1)}
On \figref{fig:psd1600} the PSDs are presented for the in- and output signals.

\fig[keepaspectratio=true,width=\textwidth]{psds30k}{Averaged periodogram for in- and output signal with 300 sample segments and N=30000}{fig:psd1600}

As seen, the power in the input signal is located around $f_{if}$ (\SI{800}{\kilo\hertz}) as expected. Similarly, the power in the output signal is located around $f_0$ (\SI{1.9}{\mega\hertz}) as expected. Since frequency changes correspond with the slope of the phase, frequency changes occur at the input. For this reason, the power will also be distributed among a range of frequencies as seen on the figure. It is also noteworthy that the power distribution is nearly identical for the in- and output. This is expected, since the point of the loop is to ``shift'' the signal from one frequency to another. The frequency vs. phase will discussed further in the last section.

\section{Time- and phase-offset (2)}
On \figref{fig:phaseuncomp1600} and \figref{fig:phaseuncomp3200}, the uncompensated in- and output phase signals are presented.

\fig[keepaspectratio=true,width=\textwidth]{phasesignals1k6}{Uncompensated in- and output phase signal for N=1600}{fig:phaseuncomp1600}
\fig[keepaspectratio=true,width=\textwidth]{phasesignals3k2}{Uncompensated in- and output phase signal for N=3200}{fig:phaseuncomp3200}

As seen, the output phase signal follows the input phase signal as wanted. A constant phase offset is present in the output signal, however, which is caused by the initial state assumptions. Since the initial state assumptions are incorrect (and not accounted for in the integral over $v(t_n)$) a constant phase offset occurs. Furthermore, the output is also delayed in respect to the input. This is caused by delay in the loop, low-pass filter and most importantly slew-rate in the NCO. The slewrate of the NCO is controlled by the NCO constant $k_{co}$, which means the NCO output takes time to adjust to input. The time-delay could be minimized by increasing $k_{co}$ but this would cause more high frequency content in the output phase.

\section{Time- and phase-errors (3)}
On \figref{fig:phasecomp} and \figref{fig:phasecomp2}, the compensated output phase signals are presented in respect to input and uncompensated output phase signals.

\fig[keepaspectratio=true,width=\textwidth]{phasesignalscomp1k6}{Compensated output phase signal for N=1600 in repect to uncompensated phase signals}{fig:phasecomp}
\fig[keepaspectratio=true,width=\textwidth]{phasesignalscomp3k2}{Compensated output phase signal for N=3200 in repect to uncompensated phase signals}{fig:phasecomp2}

As seen on the figures, the phase of the output signal follows the phase of the input rather well. Some phase-errors can be seen when the slope of the phsae changes which is caused by the NCO slew-rate discussed in the previous section. In the first samples it can also be seen that the loop has a lock-time since the initial state assumptions do not reflect the ideal initial state.

\section{Phase error and RMS error (4) and (5)}
On \figref{fig:error} and \figref{fig:error2}, the phase difference between the input and compensated output phase signals are presented.

\fig[keepaspectratio=true,width=\textwidth]{phaseerr3k}{Compensated phase error signal for N=3000}{fig:error}
\fig[keepaspectratio=true,width=\textwidth]{phaseerr30k}{Compensated phase error signal for N=30000}{fig:error2}

As seen, larger phase error occurs during the lock-time and when the slope of the phase changes. Using the error signal on \figref{fig:error2} (and omitting the first 500 samples) the RMS can be found to \SI{2.36}{\degree}. Note that the phase contains some high frequency content. This caused by the non-ideal low pass filter since it does not attenuate the high-frequency term in the expression for the $u(t_n)$-signal \eqref{eq:uneq}. These could be removed by either lowering the NCO constant or increasing the filter order. Both of these solutions could however decrease the slew-rate of the loop. In a well designed loop, the high frequency oscillations would not be present but lock-time errors cannot be removed of course.

%RMS = \\
%0.0412276521009\\
%RMS deg =\\
%2.36217046462\\

\section{Output frequency (6)}
By using the expression for the output frequency
\begin{flalign}
& & f_y(t_n) &= f_0 + k_{co} \cdot v(t_n) & &
\end{flalign}
it was plotted as presented on \figref{fig:outfreq}.

\fig[keepaspectratio=true,width=\textwidth]{outfreq}{Instantaneous output frequency}{fig:outfreq}

As seen, the output frequency also suffers from the non-ideal low pass filtering. The frequency alternates around the free running frequency as wanted, however. Furthermore, the mean of the output frequency changes consistenly with the slope of the output phase as expected.\\

Conclusively, the translational loop functions as wanted but some aspects (like phase- and time-offset) has to be accounted for. Furthermore, some improvements could be done to the loop since unwanted oscillations have been observed. It does yield an acceptable RMS phase error, however, which means it could be used as e.g. a frequency synthesizer or in an up/down-conversion stage.

%%% Local Variables: 
%%% mode: latex
%%% TeX-master: "../main.tex"
%%% End: 
