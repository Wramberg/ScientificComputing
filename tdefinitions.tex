\makeatletter
%
\def\ttransistor{\def\pst@par{}\pst@object{ttransistor}}
\def\ttransistor@i(#1){%
%  \addbefore@par{circedge=\pcangle}
  \pst@killglue
  \begingroup
  \use@par%
  \@ifnextchar({\ttransistor@iii(#1)}{\Pst@tempfalse\ttransistor@ii(#1)}}
%
\def\ttransistor@ii(#1)#2#3#4#5{% with one node, the base
  \pst@killglue%
  \ifPst@temp\pnode(#1){TBaseNode}%
  \else%
    \pst@getcoor{#1}\pst@tempA%
    \pnode(!
      \pst@tempA /YB exch \pst@number\psyunit div def
      /XB exch \pst@number\psxunit div def
      /basesep \Pst@basesep\space \pst@number\psxunit div def
      XB basesep \Pst@TRot\space cos mul add
      YB basesep \Pst@TRot\space sin mul add){TBaseNode}% base node
  \fi%
  \psdot(#1)%
  \rput[c]{\Pst@TRot}(TBaseNode){%(#1){%
    \ifPst@transistorcircle\pscircle(0.3,0){0.7}\fi%
    \ifx\psk@Ttype\pst@Ttype@FET\relax%
      \ifPst@FETmemory% atosch
        \psline[arrows=-,linewidth=\psk@I@width](-0.15,0.5)(-0.15,-0.5)%
      \fi%
      \psline[arrows=-,linewidth=\psk@I@width](TBaseNode|0,0.5)(TBaseNode|0,-0.5)%
    \else%
      \psline[arrows=-,linewidth=4\pslinewidth](TBaseNode|0,0.4)(TBaseNode|0,-0.4)%
    \fi%
    \ifnum180=\Pst@TRot\relax%
      \rput{180}(#5,0){#4} % HIER
      \ifx\psk@Ttype\pst@Ttype@FET\relax%
        \ifPst@transistorinvert\pnode(0.75,-0.5){#2}\else\pnode(0.75,-0.5){#3}\fi%
        \ifPst@transistorinvert\pnode(0.75,0.5){#3}\else\pnode(0.75,0.5){#2}\fi%
      \else%
        \ifPst@transistorinvert\pnode(0.5,-0.5){#2}\else\pnode(0.5,-0.5){#3}\fi%
        \ifPst@transistorinvert\pnode(0.5,0.5){#3}\else\pnode(0.5,0.5){#2}\fi%
      \fi%
    \else%
      \rput(#5,0){#4} % HIER
      \ifx\psk@Ttype\pst@Ttype@FET\relax%
        \ifPst@transistorinvert\pnode(0.65,0.5){#2}\else\pnode(0.65,0.5){#3}\fi%
        \ifPst@transistorinvert\pnode(0.65,-0.5){#3}\else\pnode(0.65,-0.5){#2}\fi%
      \else%
        \ifPst@transistorinvert\pnode(0.5,0.5){#2}\else\pnode(0.5,0.5){#3}\fi%
        \ifPst@transistorinvert\pnode(0.5,-0.5){#3}\else\pnode(0.5,-0.5){#2}\fi%
      \fi%
    \fi%
    \ifx\psk@Ttype\pst@Ttype@FET\relax%
      \ifnum180=\Pst@TRot\relax
        \psline[arrows=-](0.6,0.5)(0.05,0.5)
        \psline[linestyle=dashed,dash=8pt 3pt,arrows=-](0.05,0.6)(0.05,-0.6)
        \psline[arrows=-](0.05,-0.5)(0.6,-0.5)%
      \else 
        \psline[arrows=-](0.65,0.5)(0.15,0.5)
        \psline[linestyle=dashed,dash=8pt 3pt,arrows=-](0.15,0.6)(0.15,-0.6)
        \psline[arrows=-](0.15,-0.5)(0.65,-0.5)%
      \fi%
    \else%
      \psline[arrows=-](0.5,0.5)(TBaseNode)(0.5,-0.5)%
    \fi%
    \ifx\psk@Ttype\pst@Ttype@FET\relax%
%      \ifx\psk@Ttype\pst@Ttype@PNP\relax%
      \ifx\psk@FETchanneltype\pst@FETchanneltype@P\relax% Ted 2007-10-15
        \psline[arrowinset=0,arrowsize=8\pslinewidth]{->}(0.15,0)(0.65,0)%
	\qdisk(#2){1.5pt}\psline[origin={#2}]{-}(0,0.5)%
      \else%
        \psline[arrowinset=0,arrowsize=8\pslinewidth]{<-}(0.15,0)(0.65,0)%
	\qdisk(#2){1.5pt}\psline[origin={#2}]{-}(0,0.5)%
      \fi%
    \else%
      \ifx\psk@Ttype\pst@Ttype@PNP\relax%
        \psline[arrowinset=0,arrowsize=8\pslinewidth]{->}(#3)(TBaseNode)%
      \else%
         \psline[arrowinset=0,arrowsize=8\pslinewidth]{->}(TBaseNode)(#2)%
      \fi%
    \fi%
  }%
  \ifPst@temp\else\endgroup\fi%
  \ignorespaces%
}
%
\def\ttransistor@iii(#1)(#2)(#3)(#4)(#5){% with three nodes
  \pst@getcoor{#1}\pst@tempA%
  \pst@getcoor{#2}\pst@tempB%
  \pst@getcoor{#3}\pst@tempC%
  \pnode(!%
    \pst@tempA /Y1 exch \pst@number\psyunit div def
    /X1 exch \pst@number\psxunit div def
    \pst@tempB /Y2 exch \pst@number\psyunit div def
    /X2 exch \pst@number\psxunit div def
    \pst@tempC /Y3 exch \pst@number\psyunit div def
    /X3 exch \pst@number\psxunit div def
    /LR X1 X2 lt { false }{ true } ifelse def % change left-right
    /basesep \Pst@basesep\space \pst@number\psxunit div def
    /XBase X1 basesep \Pst@TRot\space cos mul add def
    /YBase Y1 basesep \Pst@TRot\space sin mul add def
    XBase YBase ){@@base}% base node
%
  \Pst@temptrue%
  \ttransistor@ii(@@base){@@emitter}{@@collector}{#4}{#5}%
%
  \if\psk@labeltransistoribase\@empty\else\psset{transistoribase=true}\fi%
  \if\psk@labeltransistoriemitter\@empty\else\psset{transistoriemitter=true}\fi%
  \if\psk@labeltransistoricollector\@empty\else\psset{transistoricollector=true}\fi%
  \ifPst@intensity\psset{transistoribase=true,transistoriemitter=true,transistoricollector=true}\fi%
%
  \bgroup\psset{style=baseOpt}\pscirc@edge(#1)(TBaseNode)\egroup%
  \ifPst@transistoribase% base current?
    \ncput[npos=0.5,nrot=\Pst@TRot]{%
      \psline[linecolor=\psk@I@color,linewidth=\psk@I@width,%
        arrowsize=6\pslinewidth,arrowinset=0]{->}(-.1,0)(.1,0)}%
    \naput[npos=0.5]{\csname\psk@I@labelcolor\endcsname\psk@labeltransistoribase}%
  \fi%
  \bgroup%
    \psset{style=collectorOpt}%
    \ifPst@transistorinvert\pscirc@edge(#3)(@@emitter)\else\pscirc@edge(#3)(@@collector)\fi%
  \egroup%
  \ncput[npos=2]{\pnode{\ifPst@transistorinvert @@emitter\else @@collector\fi}}%
  \ifPst@transistoriemitter% emitter current?
    \ifPst@transistorinvert% emitter/collector changed?
      \ncput[npos=\pscirc@edge@sector,nrot=:U]{%
        \psline[linecolor=\psk@I@color,linewidth=\psk@I@width,%
    arrowsize=6\pslinewidth,arrowinset=0]{->}(-0.1,0)(0.1,0)}
      \nbput[npos=\pscirc@edge@sector]{\csname\psk@I@labelcolor\endcsname\psk@labeltransistoriemitter}
    \fi\fi%
  \ifPst@transistoricollector% collector current?
    \ifPst@transistorinvert\else% emitter/collector changed?
      \ncput[npos=\pscirc@edge@sector,nrot=:U]{%
        \psline[linecolor=\psk@I@color,linewidth=\psk@I@width,%
    arrowsize=6\pslinewidth,arrowinset=0]{->}(-.1,0)(.1,0)}
      \nbput[npos=\pscirc@edge@sector]{\csname\psk@I@labelcolor\endcsname\psk@labeltransistoricollector}
    \fi\fi%
  \bgroup
  \psset{style=emitterOpt}
  \ifPst@transistorinvert\pscirc@edge(#2)(@@collector)\else\pscirc@edge(#2)(@@emitter)\fi
  \egroup
  \ncput[npos=2]{\pnode{\ifPst@transistorinvert @@collector\else @@emitter\fi}}
  \ifPst@transistoriemitter
    \ifPst@transistorinvert\else
      \ncput[npos=\pscirc@edge@sector,nrot=:U]{%
        \psline[linecolor=\psk@I@color,linewidth=\psk@I@width,
    arrowsize=6\pslinewidth,arrowinset=0]{<-}(-.1,0)(.1,0)}
      \naput[npos=\pscirc@edge@sector]{\csname\psk@I@labelcolor\endcsname\psk@labeltransistoriemitter}
    \fi\fi%
  \ifPst@transistoricollector% collector current?
    \ifPst@transistorinvert% emitter/collector changed?
      \ncput[npos=\pscirc@edge@sector,nrot=:U]{%
        \psline[linecolor=\psk@I@color,linewidth=\psk@I@width,
    arrowsize=6\pslinewidth,arrowinset=0]{<-}(-.1,0)(.1,0)}
      \naput[npos=\pscirc@edge@sector]{\csname\psk@I@labelcolor\endcsname\psk@labeltransistoricollector}
    \fi\fi
  \psline[linestyle=none](#1)(#2)% for the end arrows
  \psline[linestyle=none](#1)(#3)% for the end arrows
  \endgroup
  \ignorespaces%
}

\def\tnot(#1){%
	\rput(#1){%
		\pnode(0,0){notI}
	        \psline([nodesep=0.25,angle=90]notI)([nodesep=0.25,angle=-90]notI)
	        \psline([nodesep=0.25,angle=90]notI)([nodesep=0.5,angle=0]notI)
	        \psline([nodesep=0.25,angle=-90]notI)([nodesep=0.5,angle=0]notI)
	        \pscircle([nodesep=0.6,angle=0]notI){0.1}
	        }
    }
\def\tor(#1){%
    \rput(#1){%
		\pnode(0,0){or2}
        \pnode([nodesep=0.5,angle=90]or2){or1}
        \pnode([nodesep=0.25,angle=90]or2){ormid}
        \pnode([nodesep=1]ormid){or3}
        \psbezier[showpoints=false]{-}([nodesep=0.25,angle=100]or1)([nodesep=0.25,angle=5]ormid)([nodesep=0.25,angle=-5]ormid)([nodesep=0.25,angle=-100]or2)
        \psline([nodesep=0.1]or2)([nodesep=-0.5]or2)
        \psline([nodesep=0.1]or1)([nodesep=-0.5]or1)
        \psbezier[showpoints=false]{-}([nodesep=0.25,angle=100]or1)([nodesep=0.5,angle=20]or1)([nodesep=0.6,angle=10]or1)(or3)
        \psbezier[showpoints=false]{-}([nodesep=0.25,angle=-100]or2)([nodesep=0.5,angle=-20]or2)([nodesep=0.6,angle=-10]or2)(or3)
        }
    }
\def\tand(#1){%
    \rput(#1){%
        \pnode(0,0){and0}
		\pnode([nodesep=0.5,angle=90]and0){and1}
        \pnode([nodesep=0.25,angle=90]and1){and1h}
        \pnode([nodesep=0.25,angle=-90]and0){and0h}
        \pnode([nodesep=0.5,angle=-90]and1h){andm}
        \psarc([nodesep=0.5]andm){0.5}{-90}{90}
        \psline(and1h)([nodesep=0.5]and1h)
        \psline(and0h)([nodesep=0.5]and0h)
        \psline(and1)([nodesep=-0.5]and1)
        \psline(and0)([nodesep=-0.5]and0)
        \psline(and1h)(and0h)
        \psline([nodesep=1]andm)([nodesep=1.5]andm)
        }
    }



    
    

\makeatother

