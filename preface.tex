\chapter*{Preface}
This report has been written by group 871 who are 8th semester students at Signal Processing and Computing, Aalborg University. The report serves as documentation for the 8th semester project, which has been devised from February the 5th to May the 29th, 2014. The theme for this project is \textit{``Reconfigurable Computing''} where the subject \textit{``Design and Implementation of PPM-demodulator on Reconfigurable Hardware Platform''} has been chosen.

\section*{Reading instructions}
This report is addressed to supervisors, students and others who are interested in the field of Signal Processing and Computing. By extension the report presumes the reader to have a basic knowledge within this field. The report contains references following the Harvard-method i.e. [Surname,Year]. These references points to a bibliography in the back of the report. \\

Figures and tables are numbered according to their location in the report. Lists of figures and tables can also be found in the back of the report. Unless otherwise specified, the figures and tables in the report have been produced by the authors. When referring to these, their numbers will be used which will be followed by a page number if they are not on the same page.\\

Formulas and calculations are also numbered according to their location. The presentation of these are inspired by the ISO-31 standard for technical communication which, amongst other things, means that the SI-unit system is used. The SI-unit system will, as an exception, not be used when following methods described in American standards.\\
% The rounding of component values may, as an exception, not follow this standard because it is done to the nearest accessible value instead.\\

Throughout the report, references to appendices will occur. The appendices are placed in the back of the report as well as on the attached CD. In the back of the report, appendices like test descriptions can be found while datasheets and source codes are placed on the CD. All appendices are produced by the authors unless otherwise specified.

\section*{Report structure}
The report is divided into an analysis part, a design part and a test part. In the analysis part, overall system specifications are obtained which leads to a requirements specification. This specification is subsequently used to obtain subsystem requirements that can be used in the design part. In the design part, each subsystem is designed, optimized and implemented. In the test part, an acceptance test is performed to ensure that the system complies with the requirements. In \chref{ch:intro}, a more detailed report structure is presented after the project has been introduced.

%The report starts with an introduction and analysis part, where preliminary information is gathered. Subsequently, different solutions are discussed and a design choice is made from these. This is followed by a requirements specification and an appertaining subsystem division. Based on the subsystem division, the design, simulation and test of each subsystem is described. This design part is concluded with an integration of all subsystems (in python) followed by a test to verify the algorithm. Subsequently, something with the subsystems, architecture and test of this... dont know this yet... More specifically the report is structured as follows\\
%% to the project and the ADS-B service. Moreover, it contains an analysis of an existing ADS-B receiver which this project extends upon. The second part contains design considerations which describes different solutions on a block level. This part is concluded with a design choice, requirements specification and an appertaining subsystem division. The third part decribes the design, optimization and simulation of each subsystem. This part is concluded with an integration of all subsystems in a simulation framework which is used to verify the algorithm. The last part describes the design and optimization of the reconfigurable architecture and the mapping of the algorithm onto this architecture. This part is concluded with an acceptance test followed by a conclusion. More specifically the report is structured as follows:\\

%% The premilimary analysis describes the ADS-B service and the GomX-1 ADS-B receiver. This leads to a discussion regarding the GomX-1 receiver and the disadvantages it has. This is followed by design considerations regarding the GomX-2 receiver where a design choice is made in collaboration with GomSpace. awd 
%% The report starts with a preliminary analysis which describes relevant background information and obtains requirements for the system. It continues by describing and dividing the overall system into different subsystems and obtaining requirements for each of those. Subsequently it describes the design and test of each subsystem which leads to an implementation of the whole system. The system is lastly brought to bear on the initial requirements through various tests and a conclusion is drawn from these.\\
%Chapter 1 contains an \textbf{introduction} which introduces the project and ADS-B in general. Subsequently it describes a delimitation of the project in respect to the semester theme.\\

%Chapter 2 contains a \textbf{preliminary analysis} which describes ADS-B in more detail. Subsequently, this chapter also describes an analysis of an existing ADS-B receiver.\\

%Chapter 3 describes several \textbf{design considerations} which is based on the findings in chapter 2. This chapter is concluded with a design choice which is used in the remainder of the report.\\

%Chapter 4 contains the \textbf{requirements} to the overall system as well as every subsystem. The requirements to the overall system are obtained from the previous chapters, whereas the requirements for the subsystems are derived from internal interfaces.\\

%Chapters 4 to 7 describes the \textbf{design} and testing/simulation of the subsystems. Each chapter contains the design considerations and tests for a single subsystem.\\

%Chapter 8 describes the design and optimzation of the chosen architecture in respect to the algorithm. This includes simulations and tests for each subsystem as well as a complete integration on the architecture.\\

%Chapter 9 describes the \textbf{acceptance test} which examines if the system complies with the requirements specification from chapter 4. Moreover it also tests the general functionality of the algorithm.\\

% \textbf{Chapter 10} contains the \textbf{conclusion} which is based on the test-results from the previous chapters and summarizes the outcome of the project.\\

% \textbf{Chapter 11} describes \textbf{future possibilities} where alternative solutions are discussed and ideas for future development is outlined.\\

% \textbf{Chapter 12} contains the \textbf{list of references} used in the report. This includes bibliography, list of figures etc. \\

% \textbf{Chapter 13} contains an \textbf{appendix overview} which is a summary of the material used and a list of where to find it.\\

% Following chapter 13 the \textbf{appendices} are placed.\\

