%%%%%    PREAMBLE     %%%%%
% Basics
\documentclass[11pt,twoside]{report}
\usepackage{type1cm}
\usepackage[utf8]{inputenc}
\usepackage[english]{babel}
\usepackage{a4}
\usepackage{float}
\usepackage{lastpage}
\usepackage{amsmath} 
\usepackage[table]{xcolor}
\usepackage{umoline}		   % Package til \Overline{text} .. husk case sensitive !
\usepackage[labelfont=bf,small]{caption}

% \begin{matrix*}, also pmatrix*, bmatrix*, Bmatrix*, vmatrix*, Vmatrix*
\usepackage{mathtools}

% Shah function symbol
\usepackage[OT2,T1]{fontenc}
\DeclareSymbolFont{cyrletters}{OT2}{wncyr}{m}{n}
\DeclareMathSymbol{\Shah}{\mathalpha}{cyrletters}{"58}

% Package til \FloatBarrier 
\usepackage[section]{placeins} 

% Package for lorem ipsum ?
\usepackage{lipsum}

% farve til tabler
\usepackage[table]{xcolor}

% Farver
\usepackage{array}
\definecolor{figbg}{RGB}{220,220,220}
\definecolor{gray225}{RGB}{225,225,225}
\definecolor{gray230}{RGB}{230,230,230}
\definecolor{gray235}{RGB}{235,235,235}
\definecolor{lightergray}{RGB}{250,250,250}
\definecolor{codesnippetbg}{RGB}{235,240,245}
\definecolor{codesnippetbg}{RGB}{216,223,230}
\definecolor{codesnippetbg}{RGB}{225,231,237}
\definecolor{jgray}{RGB}{130,130,130}
\definecolor{darkgray}{RGB}{40,40,40}
\definecolor{darkred}{RGB}{136,0,21}
\definecolor{darkblue}{RGB}{0,4,183}
\definecolor{darkerred}{RGB}{110,0,20}
\definecolor{darkgreen}{RGB}{0,160,0}
\definecolor{darkgreen2}{RGB}{16,100,36}
\definecolor{pink}{rgb}{1.,0.75,0.8}
\definecolor{purple}{RGB}{139,21,152}
\definecolor{sand}{RGB}{245,240,220}
\definecolor{lighterishblue}{RGB}{209,229,252}

% Used in report strcuture figure
\definecolor{rblue}{RGB}{221,221,237}
\definecolor{rgreen}{RGB}{221,237,221}
\definecolor{rred}{RGB}{237,221,221}

\definecolor{aaublue}{RGB}{31,73,125}
\definecolor{aaugray}{RGB}{129,131,145}
\definecolor{verydarkblue}{RGB}{1,43,95}
\definecolor{lightblue}{RGB}{129,179,239}

% pænere tabeller
\usepackage{booktabs}

% Indhold -> Indholdsfortegnelse, Bilag -> Appendiks
% \addto\captionsdanish{
% 	\renewcommand\appendixname{Appendix}
% 	\renewcommand\contentsname{Table of contents}
% }

\usepackage{appendix}

% Fjerner mellemrum efter , i equations
% \usepackage{icomma}

% Giver mulighed for at inkludere pdf'er
%\usepackage{pdfpages}

% Overhold standarder for SI-enheder
\usepackage[%
    %decimalsymbol=comma,                   % Komma som tusindtalsseperator (not anymore!)
    per-mode=fraction,                     % Brug fraction ved fx \meter\per\second (ellers bruger den ms^{-1})
    exponent-product=\cdot,                % Brug \cdot ved videnskabelig notation, fx 21\cdot 10^{3} (ellers bruger den \times)
    complex-root-position = before-number, % Sæt kompleks-del foran tallet
    output-complex-root = \text{j},        % Brug j for kompleks notation
    %output-exponent-marker = \text{e}     % Brug 21e3 i stedet for 21\cdot 10^3
    group-digits = true,                    % Lille mellemrum som tusindtalsseperator
    binary-units = true
    ]{siunitx}

\newcommand{\SIf}[2]{ % \SI med lille fraction (god til in-line stuff)
  %\SI[fraction-function=\slfrac]{#1}{#2}
  \SI[per-mode=symbol]{#1}{#2}
}

\newcommand{\SIp}[2]{
  \SI[per-mode=symbol,per-symbol=p]{#1}{#2}
}
  

% Giver adgang til \singlespacing, \doublespacing, og \onehalfspacing
\usepackage{setspace}

% Halvanden linjeafstand (se også software.tex)
\onehalfspacing

\makeatletter
%
\def\ttransistor{\def\pst@par{}\pst@object{ttransistor}}
\def\ttransistor@i(#1){%
%  \addbefore@par{circedge=\pcangle}
  \pst@killglue
  \begingroup
  \use@par%
  \@ifnextchar({\ttransistor@iii(#1)}{\Pst@tempfalse\ttransistor@ii(#1)}}
%
\def\ttransistor@ii(#1)#2#3#4#5{% with one node, the base
  \pst@killglue%
  \ifPst@temp\pnode(#1){TBaseNode}%
  \else%
    \pst@getcoor{#1}\pst@tempA%
    \pnode(!
      \pst@tempA /YB exch \pst@number\psyunit div def
      /XB exch \pst@number\psxunit div def
      /basesep \Pst@basesep\space \pst@number\psxunit div def
      XB basesep \Pst@TRot\space cos mul add
      YB basesep \Pst@TRot\space sin mul add){TBaseNode}% base node
  \fi%
  \psdot(#1)%
  \rput[c]{\Pst@TRot}(TBaseNode){%(#1){%
    \ifPst@transistorcircle\pscircle(0.3,0){0.7}\fi%
    \ifx\psk@Ttype\pst@Ttype@FET\relax%
      \ifPst@FETmemory% atosch
        \psline[arrows=-,linewidth=\psk@I@width](-0.15,0.5)(-0.15,-0.5)%
      \fi%
      \psline[arrows=-,linewidth=\psk@I@width](TBaseNode|0,0.5)(TBaseNode|0,-0.5)%
    \else%
      \psline[arrows=-,linewidth=4\pslinewidth](TBaseNode|0,0.4)(TBaseNode|0,-0.4)%
    \fi%
    \ifnum180=\Pst@TRot\relax%
      \rput{180}(#5,0){#4} % HIER
      \ifx\psk@Ttype\pst@Ttype@FET\relax%
        \ifPst@transistorinvert\pnode(0.75,-0.5){#2}\else\pnode(0.75,-0.5){#3}\fi%
        \ifPst@transistorinvert\pnode(0.75,0.5){#3}\else\pnode(0.75,0.5){#2}\fi%
      \else%
        \ifPst@transistorinvert\pnode(0.5,-0.5){#2}\else\pnode(0.5,-0.5){#3}\fi%
        \ifPst@transistorinvert\pnode(0.5,0.5){#3}\else\pnode(0.5,0.5){#2}\fi%
      \fi%
    \else%
      \rput(#5,0){#4} % HIER
      \ifx\psk@Ttype\pst@Ttype@FET\relax%
        \ifPst@transistorinvert\pnode(0.65,0.5){#2}\else\pnode(0.65,0.5){#3}\fi%
        \ifPst@transistorinvert\pnode(0.65,-0.5){#3}\else\pnode(0.65,-0.5){#2}\fi%
      \else%
        \ifPst@transistorinvert\pnode(0.5,0.5){#2}\else\pnode(0.5,0.5){#3}\fi%
        \ifPst@transistorinvert\pnode(0.5,-0.5){#3}\else\pnode(0.5,-0.5){#2}\fi%
      \fi%
    \fi%
    \ifx\psk@Ttype\pst@Ttype@FET\relax%
      \ifnum180=\Pst@TRot\relax
        \psline[arrows=-](0.6,0.5)(0.05,0.5)
        \psline[linestyle=dashed,dash=8pt 3pt,arrows=-](0.05,0.6)(0.05,-0.6)
        \psline[arrows=-](0.05,-0.5)(0.6,-0.5)%
      \else 
        \psline[arrows=-](0.65,0.5)(0.15,0.5)
        \psline[linestyle=dashed,dash=8pt 3pt,arrows=-](0.15,0.6)(0.15,-0.6)
        \psline[arrows=-](0.15,-0.5)(0.65,-0.5)%
      \fi%
    \else%
      \psline[arrows=-](0.5,0.5)(TBaseNode)(0.5,-0.5)%
    \fi%
    \ifx\psk@Ttype\pst@Ttype@FET\relax%
%      \ifx\psk@Ttype\pst@Ttype@PNP\relax%
      \ifx\psk@FETchanneltype\pst@FETchanneltype@P\relax% Ted 2007-10-15
        \psline[arrowinset=0,arrowsize=8\pslinewidth]{->}(0.15,0)(0.65,0)%
	\qdisk(#2){1.5pt}\psline[origin={#2}]{-}(0,0.5)%
      \else%
        \psline[arrowinset=0,arrowsize=8\pslinewidth]{<-}(0.15,0)(0.65,0)%
	\qdisk(#2){1.5pt}\psline[origin={#2}]{-}(0,0.5)%
      \fi%
    \else%
      \ifx\psk@Ttype\pst@Ttype@PNP\relax%
        \psline[arrowinset=0,arrowsize=8\pslinewidth]{->}(#3)(TBaseNode)%
      \else%
         \psline[arrowinset=0,arrowsize=8\pslinewidth]{->}(TBaseNode)(#2)%
      \fi%
    \fi%
  }%
  \ifPst@temp\else\endgroup\fi%
  \ignorespaces%
}
%
\def\ttransistor@iii(#1)(#2)(#3)(#4)(#5){% with three nodes
  \pst@getcoor{#1}\pst@tempA%
  \pst@getcoor{#2}\pst@tempB%
  \pst@getcoor{#3}\pst@tempC%
  \pnode(!%
    \pst@tempA /Y1 exch \pst@number\psyunit div def
    /X1 exch \pst@number\psxunit div def
    \pst@tempB /Y2 exch \pst@number\psyunit div def
    /X2 exch \pst@number\psxunit div def
    \pst@tempC /Y3 exch \pst@number\psyunit div def
    /X3 exch \pst@number\psxunit div def
    /LR X1 X2 lt { false }{ true } ifelse def % change left-right
    /basesep \Pst@basesep\space \pst@number\psxunit div def
    /XBase X1 basesep \Pst@TRot\space cos mul add def
    /YBase Y1 basesep \Pst@TRot\space sin mul add def
    XBase YBase ){@@base}% base node
%
  \Pst@temptrue%
  \ttransistor@ii(@@base){@@emitter}{@@collector}{#4}{#5}%
%
  \if\psk@labeltransistoribase\@empty\else\psset{transistoribase=true}\fi%
  \if\psk@labeltransistoriemitter\@empty\else\psset{transistoriemitter=true}\fi%
  \if\psk@labeltransistoricollector\@empty\else\psset{transistoricollector=true}\fi%
  \ifPst@intensity\psset{transistoribase=true,transistoriemitter=true,transistoricollector=true}\fi%
%
  \bgroup\psset{style=baseOpt}\pscirc@edge(#1)(TBaseNode)\egroup%
  \ifPst@transistoribase% base current?
    \ncput[npos=0.5,nrot=\Pst@TRot]{%
      \psline[linecolor=\psk@I@color,linewidth=\psk@I@width,%
        arrowsize=6\pslinewidth,arrowinset=0]{->}(-.1,0)(.1,0)}%
    \naput[npos=0.5]{\csname\psk@I@labelcolor\endcsname\psk@labeltransistoribase}%
  \fi%
  \bgroup%
    \psset{style=collectorOpt}%
    \ifPst@transistorinvert\pscirc@edge(#3)(@@emitter)\else\pscirc@edge(#3)(@@collector)\fi%
  \egroup%
  \ncput[npos=2]{\pnode{\ifPst@transistorinvert @@emitter\else @@collector\fi}}%
  \ifPst@transistoriemitter% emitter current?
    \ifPst@transistorinvert% emitter/collector changed?
      \ncput[npos=\pscirc@edge@sector,nrot=:U]{%
        \psline[linecolor=\psk@I@color,linewidth=\psk@I@width,%
    arrowsize=6\pslinewidth,arrowinset=0]{->}(-0.1,0)(0.1,0)}
      \nbput[npos=\pscirc@edge@sector]{\csname\psk@I@labelcolor\endcsname\psk@labeltransistoriemitter}
    \fi\fi%
  \ifPst@transistoricollector% collector current?
    \ifPst@transistorinvert\else% emitter/collector changed?
      \ncput[npos=\pscirc@edge@sector,nrot=:U]{%
        \psline[linecolor=\psk@I@color,linewidth=\psk@I@width,%
    arrowsize=6\pslinewidth,arrowinset=0]{->}(-.1,0)(.1,0)}
      \nbput[npos=\pscirc@edge@sector]{\csname\psk@I@labelcolor\endcsname\psk@labeltransistoricollector}
    \fi\fi%
  \bgroup
  \psset{style=emitterOpt}
  \ifPst@transistorinvert\pscirc@edge(#2)(@@collector)\else\pscirc@edge(#2)(@@emitter)\fi
  \egroup
  \ncput[npos=2]{\pnode{\ifPst@transistorinvert @@collector\else @@emitter\fi}}
  \ifPst@transistoriemitter
    \ifPst@transistorinvert\else
      \ncput[npos=\pscirc@edge@sector,nrot=:U]{%
        \psline[linecolor=\psk@I@color,linewidth=\psk@I@width,
    arrowsize=6\pslinewidth,arrowinset=0]{<-}(-.1,0)(.1,0)}
      \naput[npos=\pscirc@edge@sector]{\csname\psk@I@labelcolor\endcsname\psk@labeltransistoriemitter}
    \fi\fi%
  \ifPst@transistoricollector% collector current?
    \ifPst@transistorinvert% emitter/collector changed?
      \ncput[npos=\pscirc@edge@sector,nrot=:U]{%
        \psline[linecolor=\psk@I@color,linewidth=\psk@I@width,
    arrowsize=6\pslinewidth,arrowinset=0]{<-}(-.1,0)(.1,0)}
      \naput[npos=\pscirc@edge@sector]{\csname\psk@I@labelcolor\endcsname\psk@labeltransistoricollector}
    \fi\fi
  \psline[linestyle=none](#1)(#2)% for the end arrows
  \psline[linestyle=none](#1)(#3)% for the end arrows
  \endgroup
  \ignorespaces%
}

\def\tnot(#1){%
	\rput(#1){%
		\pnode(0,0){notI}
	        \psline([nodesep=0.25,angle=90]notI)([nodesep=0.25,angle=-90]notI)
	        \psline([nodesep=0.25,angle=90]notI)([nodesep=0.5,angle=0]notI)
	        \psline([nodesep=0.25,angle=-90]notI)([nodesep=0.5,angle=0]notI)
	        \pscircle([nodesep=0.6,angle=0]notI){0.1}
	        }
    }
\def\tor(#1){%
    \rput(#1){%
		\pnode(0,0){or2}
        \pnode([nodesep=0.5,angle=90]or2){or1}
        \pnode([nodesep=0.25,angle=90]or2){ormid}
        \pnode([nodesep=1]ormid){or3}
        \psbezier[showpoints=false]{-}([nodesep=0.25,angle=100]or1)([nodesep=0.25,angle=5]ormid)([nodesep=0.25,angle=-5]ormid)([nodesep=0.25,angle=-100]or2)
        \psline([nodesep=0.1]or2)([nodesep=-0.5]or2)
        \psline([nodesep=0.1]or1)([nodesep=-0.5]or1)
        \psbezier[showpoints=false]{-}([nodesep=0.25,angle=100]or1)([nodesep=0.5,angle=20]or1)([nodesep=0.6,angle=10]or1)(or3)
        \psbezier[showpoints=false]{-}([nodesep=0.25,angle=-100]or2)([nodesep=0.5,angle=-20]or2)([nodesep=0.6,angle=-10]or2)(or3)
        }
    }
\def\tand(#1){%
    \rput(#1){%
        \pnode(0,0){and0}
		\pnode([nodesep=0.5,angle=90]and0){and1}
        \pnode([nodesep=0.25,angle=90]and1){and1h}
        \pnode([nodesep=0.25,angle=-90]and0){and0h}
        \pnode([nodesep=0.5,angle=-90]and1h){andm}
        \psarc([nodesep=0.5]andm){0.5}{-90}{90}
        \psline(and1h)([nodesep=0.5]and1h)
        \psline(and0h)([nodesep=0.5]and0h)
        \psline(and1)([nodesep=-0.5]and1)
        \psline(and0)([nodesep=-0.5]and0)
        \psline(and1h)(and0h)
        \psline([nodesep=1]andm)([nodesep=1.5]andm)
        }
    }



    
    

\makeatother



% Fjern indents!
\setlength{\parindent}{0in}

% Afstand mellem caption og teksten nedenunder
%\setlength{\belowcaptionskip}{10pt}
%\setlength{\textfloatsep}{5pt plus 1.0pt minus 2.0pt}
%\setlength{\floatsep}{5pt plus 1.0pt minus 2.0pt}
%\setlength{\intextsep}{5pt plus 1.0pt minus 2.0pt}
% Standard in articles:
%\textfloatsep: 20.0pt plus 2.0pt minus 4.0pt;
%\floatsep: 12.0pt plus 2.0pt minus 2.0pt;
%\intextsep: 12.0pt plus 2.0pt minus 2.0pt.
%\setlength{\abovecaptionskip}{-2ex}
%\setlength{\belowcaptionskip}{-2ex}

% Margins
\usepackage{vmargin}
%\setmargrb{2cm}{1cm}{3cm}{2cm} % Ligesom nedenstående, bare til PS
%\setmargrb{2cm}{1cm}{3cm}{2cm} % Ligesom udskrift, bare med plads til warnings i margin
\setmargrb{1.75cm}{1cm}{1.75cm}{2cm} % Optimal til udskrift! {Venstre/Ryg}{Top}{Højre/Ud}{Bund}

% Billeder
\usepackage{graphicx}
%\usepackage{epstopdf}
\graphicspath{{images/}}
\usepackage[usenames,dvipsnames]{pstricks}
\usepackage{epsfig}
\usepackage{pst-grad} % For gradients
\usepackage{pst-plot} % For axes
\usepackage{pst-circ} % For circuits
\psset{gridcolor=lightgray,subgridcolor=white,fillcolor=figbg}

% pstricks shortcuts
\newcommand{\tframe}[3][]{
  \psframe[fillcolor=figbg,fillstyle=solid,linecolor=gray#1](#2)(#3)
}
\newcommand{\tpolygon}[2][]{
  \pspolygon[fillcolor=figbg,fillstyle=solid,linecolor=gray#1]#2
}
\newcommand{\tcircle}[3][]{
  \pscircle[fillcolor=figbg,fillstyle=solid,linecolor=gray#1](#2){#3}
}

% Definer COAX stik og MOSFET til brug i pst-circ
\makeatletter
	\def\coaxr(#1){%
		\rput(#1){%
			\pscircle(0,0){0.25}%
			\pscircle[fillstyle=solid,fillcolor=black](0,0){0.07}%
		    \wire(0,0)(0.5,0)%
		    %\wire(0,-0.25)(0,-0.5)%
        }
    }
    \def\coaxl(#1){%
		\rput(#1){%
			\pscircle(0,0){0.25}%
			\pscircle[fillstyle=solid,fillcolor=black](0,0){0.07}%
		    \wire(0,0)(-0.5,0)%
		    %\wire(0,-0.25)(0,-0.5)%
        }
    }
    \def\NMOSFETR(#1){%
        \rput(#1){%
            \psline(0,0)(0.125,0)(0.125,0.375)(0.125,-0.375)%
            \psline(0.75,0.5)(0.75,0.375)(0.25,0.375)(0.25,0.5)(0.25,-0.5)(0.25,-0.375)(0.75,-0.375)(0.75,-0.5)(0.75,0)(0.5,0)%
            \pspolygon[fillstyle=solid,fillcolor=black](0.26,0)(0.5,0.125)(0.5,-0.125)%
        }
    }
    \def\NMOSFETL(#1){%
        \rput(#1){%
            \psline(0,0)(-0.125,0)(-0.125,0.375)(-0.125,-0.375)%
            \psline(-0.75,0.5)(-0.75,0.375)(-0.25,0.375)(-0.25,0.5)(-0.25,-0.5)(-0.25,-0.375)(-0.75,-0.375)(-0.75,-0.5)(-0.75,0)(-0.5,0)%
            \pspolygon[fillstyle=solid,fillcolor=black](-0.26,0)(-0.5,0.125)(-0.5,-0.125)%
        }
    }
    \def\PMOSFETR(#1){%
        \rput(#1){%
            \psline(0,0)(0.125,0)(0.125,0.375)(0.125,-0.375)%
            \psline(0.75,0)(0.75,0.5)(0.75,0.375)(0.25,0.375)(0.25,0.5)(0.25,-0.5)(0.25,-0.375)(0.75,-0.375)(0.75,-0.5)%
            \psline(0.25,0)(0.5,0)%
            \pspolygon[fillstyle=solid,fillcolor=black](0.74,0)(0.5,0.125)(0.5,-0.125)%
        }
    }
    \def\PMOSFETL(#1){%
        \rput(#1){%
            \psline(0,0)(-0.125,0)(-0.125,0.375)(-0.125,-0.375)%
            \psline(-0.75,0)(-0.75,0.5)(-0.75,0.375)(-0.25,0.375)(-0.25,0.5)(-0.25,-0.5)(-0.25,-0.375)(-0.75,-0.375)(-0.75,-0.5)%
            \psline(-0.25,0)(-0.5,0)%
            \pspolygon[fillstyle=solid,fillcolor=black](-0.74,0)(-0.5,0.125)(-0.5,-0.125)%
        }
    }
\makeatother

% References laves til links
\usepackage{hyperref}
\hypersetup{ 				% Borders omkring links fjernes
	pdfborder = {0 0 0}
}

% Pænere headers og footers
\usepackage{fancyhdr}
\setlength{\headheight}{14pt}

% Pænere chapter headings
\usepackage[Lenny]{fncychap}

\makeatletter
\renewcommand{\@chapapp}{chapter}
\makeatother
\usepackage{pst-text}
% Originale styles: ftp://cam.ctan.org/texarchive/macros/latex/contrib/fncychap/fncychap.sty
\makeatletter
  \ChNameLowerCase
  % \ChTitleLowerCase

  \renewcommand{\DOCH}{%
  }

  \renewcommand{\DOTI}[1]{%
    \vspace{-2.5cm}
    \centering
    \scalebox{0.92}{
      \begin{pspicture}(19,3)
        % \psgrid
        \pspolygon[fillstyle=solid,fillcolor=aaublue,linestyle=none](-0.3,1.05)(0.2,0.94)(19,0.94)(18.5,1.05)
%        \pspolygon[fillstyle=solid,fillcolor=aaugray,linestyle=none](17,1.05)(17.5,0.94)(18.5,0.94)(18,1.05)
        % \pspolygon[fillstyle=solid,fillcolor=aaublue,linestyle=none](18,1.05)(18.5,0.94)(19,0.94)(18.5,1.05)
%        \rput(15.73,1.53){\fontsize{70}{70} \selectfont \usefont{T1}{phv}{m}{sc} \textcolor{white}{\thechapter}}
        \rput(16.27,1.53){\fontsize{70}{70} \selectfont \usefont{T1}{phv}{m}{sc} \textcolor{lightergray}{\thechapter}}
        \rput(16.25,1.53){\fontsize{70}{70} \selectfont \usefont{T1}{phv}{m}{sc} \textcolor{aaugray}{\thechapter}}
        \rput[l](0.1,1.5){\textbf{\usefont{T1}{phv}{m}{sc} \Huge \textcolor{verydarkblue}{\FmTi{#1}}}}  
      \end{pspicture}
    }
    \vskip 25\p@
    % \hspace{1cm}
    % \Huge\sffamily\textcolor{lightblue}{{ \@chapapp\hspace{-0.12cm}}}  \\ \hspace{0.1mm}   \\
    % \CTV\textcolor{darkblue}{\fontsize{40}{40}\selectfont\FmTi{#1} \hspace{-1cm}}}
    % & \textcolor{lightblue}{\hspace{-1mm}\sffamily\fontsize{80}{100}\selectfont\thechapter} 
    % \end{tabular}
    %\ChNameLowerCase
    %\ChTitleAsIs
  }
    
  \renewcommand{\DOTIS}[1]{%
    \vspace{-2.5cm}
    \centering
    \scalebox{0.92}{
      \begin{pspicture}(19,3)
        % \psgrid
        \pspolygon[fillstyle=solid,fillcolor=aaublue,linestyle=none](-0.3,1.05)(0.2,0.94)(19,0.94)(18.5,1.05)
        \pspolygon[fillstyle=solid,fillcolor=aaublue,linestyle=none](13.5,1.05)(14,0.94)(18,0.94)(17.5,1.05)
        % \psline[linecolor=aaublue,linewidth=0.06]
        % \rput[r](19,0.55){\fontsize{19}{19} \selectfont \usefont{T1}{phv}{m}{sc} \textcolor{verydarkblue}{\@chapapp} \textcolor{verydarkblue}{\thechapter}}
        % \fontsize{30}{30} \selectfont 
        \rput[l](0.1,1.5){\textbf{\usefont{T1}{phv}{m}{sc} \Huge \textcolor{verydarkblue}{\FmTi{#1}}}}    
      \end{pspicture}
    }
    \vspace{-1.25cm}
    }
\makeatother

\usepackage{titlesec}
\titleformat{\section}
{\color{verydarkblue}\large\usefont{T1}{phv}{m}{sc}}
{\color{verydarkblue}\thesection}{0.7em}{}
\titleformat{\subsection}
{\normalfont\usefont{T1}{phv}{m}{sc}}
{\thesubsection}{0.7em}{}

\makeatletter
\renewcommand{\fnum@figure}{\usefont{T1}{phv}{m}{sc}\textcolor{verydarkblue}{\figurename~\thefigure}}
\renewcommand{\fnum@table}{\usefont{T1}{phv}{m}{sc}\textcolor{verydarkblue}{\tablename~\thetable}}
\makeatother


\titlespacing{\section}{0pt}{20pt}{5pt}

% Nemmere multirow-cells i tabeller
\usepackage{multirow}

% Nemmere multi kolonner
\usepackage{multicol}

% Matematik
\usepackage{amsmath,amsfonts,amssymb}
\newcommand{\slfrac}[2]{\left.#1\middle/#2\right.} % frac med slash
\usepackage{xfrac}

% Placer flere figurer ved siden af hinanden
% \usepackage[center]{subfigure}
\usepackage{subcaption}

% Farver på tabeller
\usepackage{colortbl}
\usepackage{array}
\definecolor{lightgray}{RGB}{220,220,220}
\definecolor{lightergray}{RGB}{250,250,250}
\definecolor{jgray}{RGB}{130,130,130}
\definecolor{darkgray}{RGB}{40,40,40}
\definecolor{darkred}{RGB}{136,0,21}
\definecolor{darkblue}{RGB}{0,4,183}
\definecolor{darkerred}{RGB}{110,0,20}
\definecolor{darkgreen}{RGB}{0,160,0}
\definecolor{darkgreen2}{RGB}{16,100,36}
\definecolor{pink}{rgb}{1.,0.75,0.8}
\definecolor{lightblue}{RGB}{153,217,234}
\definecolor{purple}{RGB}{139,21,152}
\definecolor{sand}{RGB}{245,240,220}
\definecolor{coolblue}{RGB}{0,128,255}

\renewcommand\arraystretch{1.2} % Sæt row height i tabeller

% Giver adgang til \begin{verbatimtab}[8] som viser indents i verbatim environments
\usepackage{moreverb}
\usepackage{fancyvrb}
\usepackage{relsize} % font size som \relsize{2}, relativt til alm fontsize

% Custom commands defineres
% Nemmere figurer, syntaks:
% \fig[keepaspectratio=true,height=40mm]{image}{Teksten til billledet}{billedelabel}
\newcommand{\fig}[4][width=40mm]{
	\begin{figure}[h!]
		\centering
	    \includegraphics[#1]{#2}
	    \caption{#3}
	    \label{#4}	
		%\end{centering}
	\end{figure}
      }

\newcommand{\figcap}[5][width=40mm]{
	\begin{figure}[h!]
		\centering
	    \includegraphics[#1]{#2}
	    \caption[#3]{#4}
	    \label{#5}	
		%\end{centering}
	\end{figure}
}


\newcommand{\figuc}[4][width=40mm]{
	\begin{figure}[h!]
	    \includegraphics[#1]{#2}
	    \caption{#3}
	    \label{#4}	
	\end{figure}
}

% Samme, bare til pspictures
%\pfig{
%	\begin{pspicture}(x,y)
%		<skriv billede her>
%	\end{pspicture}
%}{Caption på figure}{fig:figlabelher}
\newcommand{\pfig}[3]{
	\begin{figure}[h!]
		\centering
	    #1
	    \caption{#2}
	    \label{#3}	
	%	\end{centering}
	\end{figure}
}

\newcommand{\pcapfig}[4]{
	\begin{figure}[h!]
		\centering
	    #1
	    \caption[#2]{#3}
	    \label{#4}	
	\end{figure}
}

% Nemmere referencer, syntaks:
% \figref{labelpåfigur}         --> figur 3.4 (s. 23)
% \secref{labelpåsektion}       --> sektion 2.7 (s. 7)
% \chref{labelpåchapter}        --> kapitel 2 (s. 4)
% \tref{labelpåtabel}           --> tabel 7.4 (s. 45)
% \cref{labelpåkode}            --> kodeudsnit 8.3 (s. 76)
% \bilref{labelpåbilag}         --> bilag 2.3 (s. 25)
\usepackage{nameref} % Giver adgang til \nameref

%\newcommand{\figref}[1]{figure \ref{#1} (p. \pageref{#1})}
%\newcommand{\Figref}[1]{Figure \ref{#1} (p. \pageref{#1})}
%\newcommand{\secref}[1]{section \ref{#1} (p. \pageref{#1})}
%\newcommand{\Secref}[1]{Section \ref{#1} (p. \pageref{#1})}
%\newcommand{\chref}[1]{chapter \ref{#1} (p. \pageref{#1})}
%\newcommand{\tref}[1]{table \ref{#1} (p. \pageref{#1})}
%\newcommand{\cref}[1]{code snippet \ref{#1} (p. \pageref{#1})}
%\newcommand{\bilref}[1]{annex (Se oversigt, chap. \ref{#1} (p. \pageref{#1}))}
%\newcommand{\equref}[1]{equation \ref{#1} (p. \pageref{#1})}
%\newcommand{\apref}[1]{appendix \ref{#1} (p. \pageref{#1})}
\newcommand{\figref}[1]{figure \vref{#1}}
\newcommand{\Figref}[1]{Figure \Vref{#1}}
\newcommand{\secref}[1]{section \vref{#1}}
\newcommand{\Secref}[1]{Section \Vref{#1}}
\newcommand{\chref}[1]{chapter \ref{#1}}
\newcommand{\tref}[1]{table \vref{#1}}
\newcommand{\cref}[1]{code snippet \vref{#1}}
\newcommand{\bilref}[1]{annex (Se oversigt, chap. \ref{#1} (p. \pageref{#1}))}
\newcommand{\equref}[1]{equation \vref{#1}}
\newcommand{\apref}[1]{appendix \ref{#1}}
\newcommand{\afigref}[1]{figure \vref{#1}}
\newcommand{\aafigref}[1]{figure \ref{#1}}
\newcommand{\asecref}[1]{section \vref{#1}}
\newcommand{\atref}[1]{table \vref{#1}}
\newcommand{\acref}[1]{code snippet \vref{#1}}
\newcommand{\achref}[1]{chapter \ref{#1}}
\newcommand{\aequref}[1]{equation \vref{#1}}

%uden side tal
%\newcommand{\afigref}[1]{figure \ref{#1}}
%\newcommand{\aFigref}[1]{Figure \ref{#1}}
%\newcommand{\asecref}[1]{section \ref{#1}}
%\newcommand{\aSecref}[1]{Section \ref{#1}}
%\newcommand{\achref}[1]{chapter \ref{#1}}
%\newcommand{\atref}[1]{table \ref{#1}}
%\newcommand{\acref}[1]{code snippet \ref{#1}}
%\newcommand{\aequref}[1]{equation \ref{#1}}
%\newcommand{\aapref}[1]{appendix \ref{#1}}

% Itemize uden mellemrum mellem linjer
\newenvironment{pitemize}{
\begin{itemize}
  \setlength{\itemsep}{1pt}
  \setlength{\parskip}{0pt}
  \setlength{\parsep}{0pt}
}{\end{itemize}}

% Enumrate uden mellemrum mellem linjer
\newenvironment{penumrate}{
\begin{enumerate}
  \setlength{\itemsep}{1pt}
  \setlength{\parskip}{0pt}
  \setlength{\parsep}{0pt}
}{\end{enumerate}}

% Kommentarer i margen (noter til forfattere)
\usepackage{xkvltxp}
\usepackage[draft]{fixme}
%\usepackage{fixme}       % Skjul margin kommentarer (til udskrift)
% Nemmere kommentarer i margin, syntaks:
%  \note{DitNavn}{Din note}{teksten, noten skal hæftes ved}
% Eks.:
%  Dette er en \note{Tausen}{Hov hov! Du har vist byttet om på kage og gulerødder}{Jeg kan godt lide kage - men gulerødder er nu bedre.} haha - hilsen jesper :D
\newcommand{\note}[3]{
	\fxnote*[author=#1,footnote,nomargin]{#2}{#3}
}

% Til kode-eksempler
\usepackage{color}                        % Package til \color-kommandoen
\usepackage{listings}                     % Package til kodeeksempler
% Caption customization:
%\usepackage{xcolor}
%\usepackage{courier}
%\usepackage{caption}
%\DeclareCaptionFont{white}{\color{white}}
%\DeclareCaptionFormat{listing}{\colorbox{gray}{\parbox{\textwidth}{#1#2#3}}}
%\captionsetup[lstlisting]{format=listing,labelfont=white,textfont=white}
\newcommand{\code}[2]{
  \FloatBarrier
  \lstinputlisting[#1]{#2}
  \FloatBarrier
}

%Ny command til tabler:
%\ptable[scalebox værdi (hvis den bare skal være default skal man ikke have de firkantede parenteser med)]{søjlernes formatering}{
%  tablen uden øverste \hline
%}{caption tekst}{label tekst}
\newcommand{\ptable}[5][0.75]{
  \begin{table}[h!]
    \centering
    %  \rowcolors{2}{lightergray}{}
      \scalebox{#1}{
        \begin{tabular}{#2}
          \hline
          \rowcolor[gray]{0.8}#3
        \end{tabular}
      }
      \caption{#4}
      \label{#5}
    %\end{center}
  \end{table}
}
% Tabeller med notes under, alt andet ligesom ovenstående
% Noterne specificeres som en itemize som sidste parameter:
% \ntable[scalebox]{column format}{table uden første \hline}{caption}{label}{notes}
% ex:
% \ntable{|m{5cm}|}{
%   \textbf{Example table with notes}
%   \hline
%   \hline
%   Row 1\tnote{1} \\
%   \hline
%   Row 2\tnote{2} \\
%   \hline
%   Row 3 \\
%   \hline
% }{
%   \item[1] This is the first row
%   \item[2] And this is the second
% }
\usepackage{threeparttable}
\newcommand{\ntable}[6][0.65]{
  \begin{table}[h!]
    \centering
      \scalebox{#1}{
        \begin{threeparttable}
          \rowcolors{2}{lightergray}{}
          \begin{tabular}{#2}
          #3
          \end{tabular}
          \begin{tablenotes}
            #6
          \end{tablenotes}
        \end{threeparttable}
      }
      \caption{#4}
      \label{#5}
 %   \end{center}
  \end{table}
}

%Ny command til tabler med multirow:
%\ptablemr[scalebox værdi (hvis den bare skal være default skal man ikke have de firkantede parenteser med)]{søjlernes formatering}{
%  tablen uden øverste \hline
%}{caption tekst}{label tekst}
%Derefter skal man selv ind og sætte farve i rækkerne
\newcommand{\ptablemr}[5][0.8]{
  \begin{table}[h!]
    \centering
      \scalebox{#1}{
        \begin{tabular}{#2}
          \hline
          \rowcolor[gray]{0.8}#3
        \end{tabular}
      }
      \caption{#4}
      \label{#5}
  %  \end{center}
  \end{table}
}

%Ny command til tabler til mr. t:
%\ptable[scalebox værdi (hvis den bare skal være default skal man ikke have de firkantede parenteser med)]{søjlernes formatering}{
%  tablen 
%}{caption tekst}{label tekst}
\newcommand{\ptablemrt}[5][0.65]{
  \begin{table}[h!]
    \centering
      \rowcolors{2}{lightergray}{}
      \scalebox{#1}{
        \begin{tabular}{#2}
          #3
        \end{tabular}
      }
      \caption{#4}
      \label{#5}
 %   \end{center}
  \end{table}
}

\def\lstlistingname{Code snippet}           % Definerer hvad der står foran et stykke kodes caption
\lstset{
	basicstyle=\footnotesize\ttfamily,    % Lille skrifttype
	keywordstyle=\color{blue}\bfseries,   % Keywords blå og bold
	commentstyle=\color[RGB]{34,139,34},  % Default comments mørkegrøn
	showstringspaces=false,               % Ingen symbol for mellemrum i strings
	numbers=left,                         % Linjenumre til venstre
	numberstyle=\tiny\color{darkgray},    % Små tal på linjenumre med farve skrift
	numbersep=5pt,                        % Afstand fra linjenummer og ind til kode
	backgroundcolor=\color{lightergray},  % Bg farve
	tabsize=2,                            % Indenteringer = 4 spaces
	columns=fixed,                   	  % Kan give problemer med bredde på bogstaver men skulle ikke da vi bruger ttfamily
	breaklines=true,                      % Deler en for lang linje over to linjer
	frame=tb,                		      % Styrer hvor streger skal placeres
	captionpos=t,                         % Caption til kode under og over kodeeksemplet
	rulecolor=\color{black},			  % Farven på frame
	escapeinside={(*@}{@*)},               % Giver mulighed for at lave en (*@\label{label}@*), på en kodelinje,
%										    så man kan referere til linjen
    literate={~}{$\sim$}1 {^}{$\wedge$}1,
}
\lstdefinelanguage{bascom} {              % Definition af BASCOM-language
	classoffset=4,	
	morekeywords={\$regfile,\$crystal,Config, Output, Input, Timer1, Timer0, Do, Loop, If, Then, End, Sub, Wait, Waitms, Declare, As, Int, Word, Byte, Call, And, Or, Else, Until, Goto, Alias, Dim},
		keywordstyle=\color[RGB]{0,0,128}\bfseries,classoffset=3,
	morekeywords={PORTA, PORTB, TCCR1A, TCCR1B, TCCR1C, TCCR1D, TCNT1, OCR1A, OCR1B, OCR1C, OCR1D, PINA, PINB, PINC},
		keywordstyle=\color[RGB]{128,0,0},classoffset=2,
	morekeywords={=, \&, <, >, +, -, *, /, .},
		keywordstyle=\color[RGB]{255,0,0},classoffset=1,
	sensitive=false,
	morecomment=[l]{'},
	commentstyle=\color[RGB]{34,139,34}
}
\lstdefinelanguage{arduino} {              % Definition af ARDUINO-language
	classoffset=4,
	morekeywords={char,int,void,long,
			pinMode,random,available,read,print,millis,digitalWrite,digitalRead,analogWrite,analogRead,delay,
			for,while,switch,break,if,else,bitRead,print,println,begin,return,float,cos,sin,pow,sqrt,bitClear,bitSet,
			true,false},
		keywordstyle=\color[RGB]{204,102,0},classoffset=3,
	morekeywords={setup,Serial,Serial1,Serial2,loop},
		keywordstyle=\color[RGB]{204,102,0}\bfseries,classoffset=2,
	morekeywords={HIGH,LOW,OUTPUT,INPUT},
		keywordstyle=\color[RGB]{0,102,153},classoffset=1,
	sensitive=false,
	morecomment=[l]{//},
	stringstyle=\color[RGB]{0,0,255},
	morestring=[b]",
	morestring=[b]',
	commentstyle=\color[RGB]{50,50,50}
}
\lstdefinelanguage{csharp} {              % Definition af C#-language
	classoffset=3,
	% Variabletypes, keywords
	morekeywords={char,int,void,long,object,string,true,char,false,using,class,
			this,delegate,partial,namespace,
			for,while,switch,break,if,else,new,try,catch,private,public,\#region,\#endregion},
		keywordstyle=\color[RGB]{0,0,255},classoffset=2,
	% Classes
	morekeywords={EventArgs,SerialPortController,StringBuilder,SerialDataReceivedEventHandler,MySqlConnection,
			Exception,MySqlCommand,MySqlDataReader,ConnectionState,PointPairList,ZedGraphControl,Color,
			SymbolType,GraphPane,LineItem,Fill,Size,Point,MethodInvoker,Convert,FormClosedEventArgs,EventArgs,
			SerialPort,StopBits,Parity,Form,Form1},
		keywordstyle=\color[RGB]{43,145,175},classoffset=1,
	sensitive=true,
	morecomment=[l]{//},
	stringstyle=\color[RGB]{163,21,21},
	morestring=[b]",
	morestring=[b]',
	commentstyle=\color[RGB]{0,128,0}
}
\lstdefinelanguage{vhdl} {              % Definition af VHDL-language
	classoffset=4,
	morekeywords={library,use,all,entity,generic,map,architecture,of,is,downto,others,then,if,port,signal,elsif,else,when,for,while,
	              end,process,begin,or,and,not,xor,\&,constant,wait,for,nor,nand,in,out,type,array,with,select,case},
		keywordstyle=\color[RGB]{255,0,0},classoffset=3,
	morekeywords={std_logic,std_logic_vector,to_integer,unsigned,integer,time,natural,1,2,3,4,5,6,7,8,9,0},
		keywordstyle=\color[RGB]{0,145,185},classoffset=2,
    morekeywords={=, <, >, +, -, *, /, .},
		keywordstyle=\color[RGB]{0,0,255},classoffset=1,
	sensitive=true,
	morecomment=[l]{--},
	stringstyle=\color[RGB]{0,145,185},
	morestring=[b]",
	commentstyle=\color[RGB]{0,128,0}
}
\lstdefinelanguage{psm} {              % Definition af PSM(ASM)-language
	classoffset=4,
	morekeywords={and,or,xor,out,in,load,sub,add,comp,call,jump,ret,store,fetch,test,sr0,sr1,sl0,sl1},
		keywordstyle=\color[RGB]{0,0,255},classoffset=3,
	morekeywords={equ,dsout,dsin},
		keywordstyle=\color[RGB]{108,0,117},classoffset=2,
    morekeywords={\$,0,1,2,3,4,5,6,7,8,9},
		keywordstyle=\color[RGB]{0,232,116},classoffset=1,
	sensitive=true,
	morecomment=[l]{;},
	stringstyle=\color[RGB]{0,145,185}
%	morecomment=[l]{\$},
%    commentstyle=\color[RGB]{0,232,116}
      }
      \lstdefinelanguage%
      [bfin]{Assembler}     % add a "bfin" dialect of Assembler
      [x86masm]{Assembler}  % based on the "x86masm" dialect
      % with these extra keywords:
      {%
        morekeywords={r0,r1,r2,r3,r4,r5,p0,p1,p2,p3,p4,p5,a0,a1,%
          r0.l,r0.h,r1.l,r1.h,r2.l,r2.h,r3.l,r3.h,r4.l,r4.h,r5.l,r5.h,%
          p0.l,p0.h,p1.l,p1.h,p2.l,p2.h,p3.l,p3.h,p4.l,p4.h,p5.l,p5.h
        },%
        morecomment=[l]{\#}%
       } % etc.
      
% TOC settings
\setcounter{tocdepth}{1} % Begræns table of contents til kun 2 niveauer
%\setcounter{tocdepth}{10}
\addtocontents{toc}{\protect\thispagestyle{empty}} % Fjern sidetal fra TOC

% Flowcharts
\usepackage{tikz} 
\newcommand*\circled[1]{\tikz[baseline=(char.base)]{ % circle om tekst
            \node[shape=circle,draw,inner sep=2pt] (char) {#1};}}
\usetikzlibrary{shapes,arrows}
% Define block styles
\tikzstyle{decision} = [diamond, draw=gray, aspect=2, fill=lightgray, text width=2cm, text badly centered, node distance=3cm, inner sep=0pt,minimum height=1cm]
%\tikzstyle{decision} = [regular polygon, regular polygon sides=4, shape border rotate=45, draw, fill=lightgray, text width=1.5cm, text badly centered, node distance=3cm, inner sep=0pt,minimum height=1cm]
\tikzstyle{block} = [rectangle, draw=gray, fill=lightgray, text width=3cm, text centered, minimum height=1cm]
\tikzstyle{endpoint} = [ellipse, draw=gray, fill=lightgray, text width=2cm, text centered, minimum height=1cm]
\tikzstyle{line} = [draw, thick, -latex]

\newcommand{\flowcirc}[1]{
  \psovalbox[fillstyle=solid,fillcolor=lightgray,linecolor=gray]{#1}
}
\newcommand{\flowbox}[1]{
  \psframebox[fillstyle=solid,fillcolor=lightgray,linecolor=gray]{#1}
}
\newcommand{\flowdiamond}[1]{
  \psdiabox[fillstyle=solid,fillcolor=lightgray,linecolor=gray]{#1}
}
\usepackage{bibtopic}

\usepackage{varioref}

\makeatletter
\vref@addto\extrasenglish{%
  \def\reftextfaceafter{(page~\thevpagerefnum)}
  \def\reftextfacebefore{(page~\thevpagerefnum)}
  \def\reftextafter{(page~\thevpagerefnum)}
  \def\reftextbefore{(page~\thevpagerefnum)}
	\def\reftextfaraway#1{(page~\thevpagerefnum)}
  \def\reftextcurrent{}
}
\makeatother

% This should make latex try to avoid orphans more... Max is 10000 but will probably fuck up latex :P
%\widowpenalty=300
%\clubpenalty=300
